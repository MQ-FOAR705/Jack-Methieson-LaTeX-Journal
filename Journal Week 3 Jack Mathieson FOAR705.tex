\documentclass{article}

\begin{document}
This is a test-run of LaTex for the week 3 journal required for this Friday's Digital Humanities Class. Not sure how this is going to turn out but hey, worse comes to worse the system will completely crash and I will have divine confirmation that the problem is not with me.

It's probably not healthy to be so down on this class. Taking it a bit like a lesson in martial arts rather than skydiving might help. It could go wrong, theoretically, but the result of finishing it will be a well-honed and useful skill rather than a thrill that I never want to do again in my life.

And besides, I \textit{could} be messing up a lesson in sky-diving rather than messing up a lesson on typesetting.

Enough personal thoughts for now. Time to jump.

\textbf{Response to "messy data"}

The biggest issue that I could find with the messy data from Mozambique and Tanzania was the presentation of qualitative data in quantitative terms. Computers (and tired people) can only cope well with simple qualitative data and prefer to keep it as quantitative as possible, as numbers are easier to follow and plot than peculiar descriptions like "the building has 5 rooms if you also include the cow-shed". Also, when I tried to organise the data (because the instruction was a little unclear on us not having to do that) the recommendation that the categories be put at the top while the numeric values be listed down made for a spreadsheet that was very wide and a bit difficult to read, so while I get the format benefit I personally preferred it the other way round, where the categories were listed vertically and the values then spread horizontally away from them.
\end{document}