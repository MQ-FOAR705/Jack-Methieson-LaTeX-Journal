\documentclass{article}

\begin{document}
\textbf{Jack Mathieson Week 2 Journal}

15/08/19

This is a test-run of LaTex for the week 2 journal required for this Friday's Digital Humanities Class. Not sure how this is going to turn out but hey, worse comes to worse the system will completely crash and I will have divine confirmation that the problem is not with me.

It's probably not healthy to be so down on this class. Taking it a bit like a lesson in martial arts rather than skydiving might help. It could go wrong, theoretically, but the result of finishing it will be a well-honed and useful skill rather than a thrill that I never want to do again in my life.

And besides, I \textit{could} be messing up a lesson in sky-diving rather than messing up a lesson on typesetting.

Enough personal thoughts for now. Time to jump.

\textbf{Response to "messy data"}

The biggest issue that I could find with the messy data from Mozambique and Tanzania was the presentation of qualitative data in quantitative terms. Computers (and tired people) can only cope well with simple qualitative data and prefer to keep it as quantitative as possible, as numbers are easier to follow and plot than peculiar descriptions like "the building has 5 rooms if you also include the cow-shed". Also, when I tried to organise the data (because the instruction was a little unclear on us not having to do that) the recommendation that the categories be put at the top while the numeric values be listed down made for a spreadsheet that was very wide and a bit difficult to read, so while I get the format benefit I personally preferred it the other way round, where the categories were listed vertically and the values then spread horizontally away from them.

I'm about to attempt another submission to GitHub and add the following paragraphs... we'll see how this goes.

\textbf{Week 2 Homework (completed 15/08/19)}

\textbf{Response to Clean Data}

The data in the second Excel spreadsheet was organized much better, however there was not enough contextual data to tell me what the \textit{entire} spreadsheet was about. I did not understand at a glance what purpose the data was for. There was not enough metadata.

\textbf{Examples of data problems in your discipline}

One example of cleaning up confusing data is a project that I had to assist on in a past PACE unit in 2018 where I worked with the Macleay Museum at Sydney University. I was instructed to sort through hundreds of photos of arrows from various regions in a country so that they could be documented. Several problems immediately arose: the photos were not always clear, there was regularly data missing on some arrows, and over all this was the very inefficient system of having to save various photos to different files without the us of any special kind of software. 

The main issue was one that (I suspect) Anthropologists face regularly: they are constantly having to deal with qualitative data and translate it to quantitative data. Categorizing where the arrows came from was not always as simple as naming a region and plotting how many arrows came from there, because sometimes it was unknown if the arrow came from one place or another and so it was simply from the general area of both regions. Plotting such unspecific data, or data that does not fit with the initial constructed categories is a main challenge facing Anthropologists.

Categorizing ethnicity is another study in Anthropology that does not translate well to a numeric value. In Pierre Bourdieu's work on "\textit{The Algerians}" he begins by noting that to talk about any one group of people one needs to take the others into account also (1958). Saying that the Shawia have a population of \textit{x} and that the Kabyles have a population of \textit{y} is not so simple even if everyone is counted, because the two groups of people are not necessarily mutually exclusive, nor do the categories "Kabyle" and "Shawia" account for all the distinct (and indistinct) groups in the region. Even deciding to categorise things by region incurs a value statement on the whole analysis.
\end{document}